
%%%%%%%%%%%%%%%%%%%%%% Feynman diagram for T2bbHH

\documentclass{article}

\usepackage{feynmp-auto}

%%%%%% Custom commands
\def\chiz{$\mathrm{\widetilde{\chi}^0_1}$}
\def\chitz{$\mathrm{\widetilde{\chi}^0_{2}}$}
\def\chipm{$\mathrm{\widetilde{\chi}^\pm_{1}}$}
\def\sGlu{$\mathrm{\widetilde{g}}$}
\def\sQ{$\mathrm{\widetilde{q}}$}
\def\sQb{$\mathrm{\overline{\widetilde{q}}}$}
\def\sBot{$\mathrm{\widetilde{b}_1}$}
\def\sBotb{$\mathrm{\overline{\widetilde{b}}_1}$}
\def\sTop{$\mathrm{\widetilde{t}_1}$}
\def\sTopb{$\mathrm{\overline{\widetilde{t}}_1}$}
\def\sGra{$\mathrm{\widetilde{G}}$}
\def\PH{$\mathrm{H}$}
\def\PZ{$\mathrm{Z}$}
\def\PW{$\mathrm{W^\pm}$}
\def\Ph{$\mathrm{h}$}



\def\MainQuark{b}


%%%%%%%%%%%%%%%%%%%%%%%%%%% Document %%%%%%%%%%%%%%%%%%%%%%%%%%%
\begin{document}
\thispagestyle{empty}


%%%%%%%% THE NAME OF THE fmffile HAS TO BE ``Feynman<filename>'' TO USE compile.py %%%%%%%%%%%%%%%
\begin{fmffile}{FeynmanTChiNG}
\parbox{300mm}{

\begin{fmfgraph*}(180,90) %\fmfpen{thick}
  \fmfset{arrow_len}{cm}\fmfset{arrow_ang}{0}
  
  %%%%%%%%%%%% Specifying number of inputs/outputs
  \fmfleftn{i}{2}
  \fmfrightn{o}{8}
  \fmflabel{}{i1}
  \fmflabel{}{i2}

  \fmfforce{0.65w,1.0h}{o8}
  \fmfforce{0.75w,1.0h}{o7}
  \fmfforce{0.75w,0.0h}{o2}
  \fmfforce{0.65w,0.0h}{o1}
  \fmfforce{1.0w,1.0h}{o6}
  \fmfforce{1.0w,0.0h}{o3}
    
  %%%%%%%%%%%% Incoming protons (one line)
  \fmf{fermion,  tension=2., lab=p, label.side=right}{v1,i1}
  \fmf{fermion,  tension=2., lab=p, label.side=left}{v1,i2}
           
  %%%%%%%%%%%% Produced SUSY particles
  \fmf{dots, tension=1.5, label=\chipm, label.side=left}{v1,v3}
  \fmf{dots, tension=1.5, label=\chimp, label.side=right}{v1,v2}

  %%%%%%%%%%%%% Decays and vertex circles

  %%% Top vertex
  \fmf{fermion}{v3,o8}
  \fmflabel{soft}{o8}
  \fmf{fermion}{v3,o7}
%  \fmflabel{$\ell$}{o7}


  \fmf{dots, label=\chiz,label.dist=+3, label.side=right}{v3,v5}
  \fmf{photon}{v5,o6}
  \fmflabel{$\gamma$}{o6}
  
  \fmf{dbl_dots}{v5,o5}
  \fmflabel{\sGra}{o5}
  
  %%% Bottom vertex
  \fmf{fermion}{v2,o1}
  \fmflabel{soft}{o1}
  \fmf{fermion}{v2,o2}
%  \fmflabel{$\ell$}{o2}

  \fmf{dots, label=\chiz, label.dist=+3, label.side=left}{v2,v4}
  \fmf{fermion}{v4,o3}
  \fmflabel{Z/H}{o3}
  
  \fmf{dbl_dots}{v4,o4}
  \fmflabel{\sGra}{o4}
               
  %% Vertex circles
  \fmfdot{v2,v3,v4,v5}
           
  %%%%%%%%%%%% Additional lines on incoming protons and blob
             
  %%%%%%%%%%%% Additional lines on incoming protons and blob
  \fmffreeze
  \renewcommand{\P}[3]{\fmfi{plain}{%
      vpath(__#1,__#2) shifted (thick*(#3))}}
  \P{i1}{v1}{1.2,-1.2}
  \P{i1}{v1}{-1.2,1.2}
  \P{i2}{v1}{1.2,1.2}
  \P{i2}{v1}{-1.2,-1.2}
  \fmfv{decor.shape=circle,decor.filled=30, decor.size=.12w}{v1}



\end{fmfgraph*}
       
}           
\end{fmffile} 

\end{document}
