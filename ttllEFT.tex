
%%%%%%%%%%%%%%%%%%%%%% Feynman diagram for TChiHH

\documentclass{article}

\usepackage{feynmp-auto}

%%%%%% Custom commands
\def\chiz{$\mathrm{\widetilde{\chi}^0_1}$}
\def\chitz{$\mathrm{\widetilde{\chi}^0_{2}}$}
\def\chipm{$\mathrm{\widetilde{\chi}^\pm_{1}}$}
\def\sGlu{$\mathrm{\widetilde{g}}$}
\def\sQ{$\mathrm{\widetilde{q}}$}
\def\sQb{$\mathrm{\overline{\widetilde{q}}}$}
\def\sBot{$\mathrm{\widetilde{b}_1}$}
\def\sBotb{$\mathrm{\overline{\widetilde{b}}_1}$}
\def\sTop{$\mathrm{\widetilde{t}_1}$}
\def\sTopb{$\mathrm{\overline{\widetilde{t}}_1}$}
\def\sGra{$\mathrm{\widetilde{G}}$}
\def\PH{$\mathrm{H}$}
\def\PZ{$\mathrm{Z}$}
\def\PW{$\mathrm{W^\pm}$}
\def\Ph{$\mathrm{h}$}




%%%%%%%%%%%%%%%%%%%%%%%%%%% Document %%%%%%%%%%%%%%%%%%%%%%%%%%%
\begin{document}
\thispagestyle{empty}


%%%%%%%% THE NAME OF THE fmffile HAS TO BE ``Feynman<filename>'' TO USE compile.py %%%%%%%%%%%%%%%
\begin{fmffile}{FeynmanttllEFT}
\parbox{300mm}{

\begin{fmfgraph*}(180,90) %\fmfpen{thick}
%  \fmfset{arrow_len}{cm}\fmfset{arrow_ang}{0}
  
  %%%%%%%%%%%% Specifying number of inputs/outputs
  \fmfleftn{i}{2}
  \fmfrightn{o}{4}
  \fmflabel{g}{i1}
  \fmflabel{g}{i2}
  \fmflabel{$\mathrm{t}$}{o3}
  \fmflabel{$\bar{\mathrm{t}}$}{o4}
  \fmflabel{$\ell$}{o1}
  \fmflabel{$\bar{\ell}$}{o2}

    
  \fmf{gluon,  tension=2.}{i1,v1}
  \fmf{gluon,  tension=2.}{i2,v1}
  \fmf{gluon,  tension=2.}{v1,v2}

  \fmf{fermion,  tension=1.}{o4,v2,v3,o1}
  \fmf{fermion,  tension=1.}{o2,v3,o3}


  \fmfv{decor.shape=circle, decor.size=.04w}{v3}

\end{fmfgraph*}
       
}           
\end{fmffile} 

\end{document}
