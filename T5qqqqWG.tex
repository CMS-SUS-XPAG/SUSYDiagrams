
%%%%%%%%%%%%%%%%%%%%%% Feynman diagram for T5qqqqWG

\documentclass{article}

\usepackage{feynmp-auto}

%%%%%% Custom commands
\def\chiz{$\mathrm{\widetilde{\chi}^0_1}$}
\def\chitz{$\mathrm{\widetilde{\chi}^0_{2}}$}
\def\chipm{$\mathrm{\widetilde{\chi}^\pm_{1}}$}
\def\sGlu{$\mathrm{\widetilde{g}}$}
\def\sQ{$\mathrm{\widetilde{q}}$}
\def\sQb{$\mathrm{\overline{\widetilde{q}}}$}
\def\sBot{$\mathrm{\widetilde{b}_1}$}
\def\sBotb{$\mathrm{\overline{\widetilde{b}}_1}$}
\def\sTop{$\mathrm{\widetilde{t}_1}$}
\def\sTopb{$\mathrm{\overline{\widetilde{t}}_1}$}
\def\sGra{$\mathrm{\widetilde{G}}$}
\def\PH{$\mathrm{H}$}
\def\PZ{$\mathrm{Z}$}
\def\PW{$\mathrm{W^\pm}$}
\def\Ph{$\mathrm{h}$}




%%%%%%%%%%%%%%%%%%%%%%%%%%% Document %%%%%%%%%%%%%%%%%%%%%%%%%%%
\begin{document}
\thispagestyle{empty}


%%%%%%%% THE NAME OF THE fmffile HAS TO BE ``Feynman<filename>'' TO USE compile.py %%%%%%%%%%%%%%%
\begin{fmffile}{FeynmanT5qqqqWG}
\parbox{300mm}{

\begin{fmfgraph*}(180,90) %\fmfpen{thick}

  \fmfset{arrow_len}{cm}\fmfset{arrow_ang}{0}

  %%%%%%%%%%%% Specifying number of inputs/outputs
  \fmfleftn{i}{2}
  \fmfrightn{o}{8}
  \fmflabel{}{i1}
  \fmflabel{}{i2}

  %%%%%%%%%%%% Incoming protons (one line)
  \fmf{fermion, tension=2, lab=p, label.side=right}{v1,i1}
  \fmf{fermion, tension=2, lab=p, label.side=left}{v1,i2}

  %%%%%%%%%%%% Produced SUSY particles
  \fmf{gluon, label=\sGlu, label.side=left, label.dist=8.}{v1,v3}
  \fmf{gluon, label=\sGlu, label.side=right, label.dist=10.}{v1,v2}
  \fmf{fermion}{v1,v2}
  \fmf{fermion}{v1,v3}

  %%%%%%%%%%%%% Decays and vertex circles
  \fmf{fermion}{v3,o8}
  \fmf{fermion}{v3,o7}
  \fmflabel{q}{o8}
  \fmflabel{\anti{q}}{o7}

  \fmf{dots, label=\chiz, label.dist=3.}{v3,v5}
  \fmf{photon}{v5,o6}
  \fmflabel{$\gamma$}{o6}

  \fmf{dots}{v5,o5}
  \fmflabel{\sGra}{o5}

  \fmf{fermion}{v2,o1}
  \fmf{fermion}{v2,o2}

  \fmf{dots, label=\chipm, label.side=left, label.dist=3.}{v2,v4}

  \fmf{dots}{v4,o4}
  \fmflabel{\sGra}{o4}

  \fmf{photon}{v4,o3}
  \fmflabel{\PW}{o3}

  \fmflabel{q}{o1}
  \fmflabel{\anti{q}}{o2}

  %% Vertex circles
  \fmfdot{v2,v3,v4,v5}

  %%%%%%%%%%%% Additional lines on incoming protons and blob
             
  %%%%%%%%%%%% Additional lines on incoming protons and blob
  \fmffreeze
  \renewcommand{\P}[3]{\fmfi{plain}{%
      vpath(__#1,__#2) shifted (thick*(#3))}}
  \P{i1}{v1}{1.2,-1.2}
  \P{i1}{v1}{-1.2,1.2}
  \P{i2}{v1}{1.2,1.2}
  \P{i2}{v1}{-1.2,-1.2}
  \fmfv{decor.shape=circle,decor.filled=30, decor.size=.12w}{v1}




\end{fmfgraph*}

}
\end{fmffile}

\end{document}
